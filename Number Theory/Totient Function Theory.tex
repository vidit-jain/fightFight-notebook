\subsection*{Application in Euler's theorem}

The most famous and important property of Euler's totient function is expressed in **Euler's theorem**: \\
$$a^{\phi(m)} \equiv 1 \pmod m$$ if $a$ and $m$ are relatively prime.\\

In the particular case when $m$ is prime, Euler's theorem turns into **Fermat's little theorem**:\\
$$a^{m - 1} \equiv 1 \pmod m$$\\

Euler's theorem and Euler's totient function occur quite often in practical applications, for example both are used to compute the modular multiplicative inverse\\

As immediate consequence we also get the equivalence:\\
$$a^n \equiv a^{n \bmod \phi(m)} \pmod m$$\\
This allows computing $x^n \bmod m$ for very big $n$, especially if $n$ is the result of another computation, as it allows to compute $n$ under a modulo.\\

\subsection*{Generalization}

There is a less known version of the last equivalence, that allows computing $x^n \bmod m$ efficiently for not coprime $x$ and $m$.\\
For arbitrary $x, m$ and $n \geq \log_2 m$:\\
$$x^{n}\equiv x^{\phi(m)+[n \bmod \phi(m)]} \mod m$$\\

\subsection*{Proof:}

Let $p_1, \dots, p_t$ be common prime divisors of $x$ and $m$, and $k_i$ their exponents in $m$.\\
With those we define $a = p_1^{k_1} \dots p_t^{k_t}$, which makes $\frac{m}{a}$ coprime to $x$.\\
And let $k$ be the smallest number such that $a$ divides $x^k$.\\
Assuming $n \ge k$, we can write:\\

$$\begin{align}x^n \bmod m &= \frac{x^k}{a}ax^{n-k}\bmod m \\\\
&= \frac{x^k}{a}\left(ax^{n-k}\bmod m\right) \bmod m \\\\
&= \frac{x^k}{a}\left(ax^{n-k}\bmod a \frac{m}{a}\right) \bmod m \\\\
&=\frac{x^k}{a} a \left(x^{n-k} \bmod \frac{m}{a}\right)\bmod m \\\\
&= x^k\left(x^{n-k} \bmod \frac{m}{a}\right)\bmod m
\end{align}$$

The equivalence between the third and forth line follows from the fact that $ab \bmod ac = a(b \bmod c)$.
Indeed if $b = cd + r$ with $r < c$, then $ab = acd + ar$ with $ar < ac$.

Since $x$ and $\frac{m}{a}$ are coprime, we can apply Euler's theorem and get the efficient (since $k$ is very small; in fact $k \le \log_2 m$) formula:
$$x^n \bmod m = x^k\left(x^{n-k \bmod \phi(\frac{m}{a})} \bmod \frac{m}{a}\right)\bmod m.$$

This formula is difficult to apply, but we can use it to analyze the behavior of $x^n \bmod m$. We can see that the sequence of powers $(x^1 \bmod m, x^2 \bmod m, x^3 \bmod m, \dots)$ enters a cycle of length $\phi\left(\frac{m}{a}\right)$ after the first $k$ (or less) elements. \\
$\phi\left(\frac{m}{a}\right)$ divides $\phi(m)$ (because $a$ and $\frac{m}{a}$ are coprime we have $\phi(a) \cdot \phi\left(\frac{m}{a}\right) = \phi(m)$), therefore we can also say that the period has length $\phi(m)$.\\
And since $\phi(m) \ge \log_2 m \ge k$, we can conclude the desired, much simpler, formula:\\
$$ x^n \equiv x^{\phi(m)} x^{(n - \phi(m)) \bmod \phi(m)} \bmod m \equiv x^{\phi(m)+[n \bmod \phi(m)]} \mod m.$$

